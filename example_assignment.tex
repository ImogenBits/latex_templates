\documentclass{assignment}
% by default the header is printed in english, the 'german' option will switch it to german.
% adding the 'pointsbox' option will include a text field where your TA can enter your point total.

% course name, group number, and assignment number can optionally be specified with the respective commands
\course{Boring Math}
\group{123}
\assignment{1}

% group members are added via the \member{full name}{matriculation number} command
\member{Ada Lovelace}{486921}
\member{Emmy Noether}{077734}
\member{Margaret Hamilton}{160769}


% the commondefns package includes package imports and macro definitions I often use.
% by default it only does slight improvements to text layout and includes commonly used math libraries.
% you can turn them off with 'notext' and/or 'nomath'. when using the math parts, it will change the definitions of
% some commonly disliked variable letter renderings, if you want to keep the default ones you can pass 'keepvarletters'.
% the option 'graphs' imports tikz, commonly used libraries, and defines generally useful styles.
% 'logic' and 'complexity' contain macros that are useful when working in the respective fields.
% 'rwthcolors' contains color defns for the specific shades used by the rwth
\usepackage[logic, complexity, graphs, rwthcolors]{commondefns}

\begin{document}
    \section{}
    \subsection{}
    This is our Assignment!

    \subsubsection{}
    We can now use pretty versions of $\phi$ and $\epsilon$, the round verion of $\phi$ is also available: $\ophi$.

    Common logic symbols are easy to use:\\
    If $\FO \supseteq \Phi \models \A$ for some infinite \A, then for every cardinal $\kappa \geq |\A|$ there exists a $\B \models \Phi$ with $|\B| \geq \kappa$.

    And the same is true for complexity classes:\\
    For any function $f(n) \in \Omega(\log n)$,\\
    $\NSPACE(f(n)) = \co\NSPACE(f(n))$ and $\NSPACE(f(n)) \subseteq \SPACE(f(n)^2)$.

    We can also make some graphs:\\
    \begin{tikzpicture}[nodes=outline]
        \node (1) at (0, 0) {a};
        \node (2) at (1, 0) {b};
        \node (3) at (1.5, -1) {c};
        \node (4) at (2, -0.5) {d};
        \node (5) at (0.5, -1) {e};
        \draw (1) to (2);
        \draw (2) to (3);
        \draw (3) to (1);
        \draw (3) to (4);
        \draw (5) to (1);
        \draw (5) to (3);
    \end{tikzpicture},
    \begin{tikzpicture}[nodes=letters, nodes=rwth-magenta]
        \node (1) at (0, 0) {a};
        \node (2) at (0.5, -0.65) {b};
        \node (3) at (1.5, -1) {c};
        \node (4) at (2, -0.5) {d};
        \node (5) at (0, -1) {e};
        \draw (1) to (2);
        \draw (2) to (3);
        \draw (3) to (1);
        \draw (3) to (4);
        \draw (5) to (1);
        \draw (5) to (3);
    \end{tikzpicture},
    \begin{tikzpicture}[nodes=dot, nodes=rwth-blue]
        \node (1) at (0, 0) {};
        \node (2) at (0.5, -0.65) {};
        \node (3) at (1.5, -1) {};
        \node (5) at (0, -1) {};
        \draw (1) to (2);
        \draw (2) to (3);
        \draw (3) to (1);
        \draw[snake] (3) to[bend right] (1);
        \draw[->] (5) to (1);
        \draw[->>] (3) to (5);
    \end{tikzpicture}
\end{document}